\documentclass[a4paper,12pt]{article}
\usepackage[hidelinks]{hyperref}
\usepackage{graphicx}
\usepackage{float}
\usepackage{caption}
\begin{document}
\begin{center}

%Cover page
\Huge\textbf{Software Architecture Documentation(uWatch digital forensic tool)\\}
																											
\vspace{2 cm}

\LARGE\textbf{Group Name:} MPHETamines\newline
 
 
 
 
 
\vspace{0.5 cm}
\begin{tabular}{lr}
Taariq Ghoord&10132806
\\ 
Martha Mohlala&10353403
\\
Phethile Mkhabela&12097561
\\
Sboniso Masilela&10416260
\\
Harrison Maphuti Setati&12310043\\
\end{tabular}

\vspace{1cm}
\textbf{Git repository link:\\}
\url{https://github.com/MPHETamines/MPHETamines/}

\vspace{1cm}
\textbf{Date:} 31 July 2015
\end{center}
\pagenumbering{gobble}
\newpage

%table of contents
\tableofcontents

\newpage
\pagenumbering{arabic}

\section{Introduction}
This section deals with the software architecture requirements of the uWatch
system being designed. It handles all aspects of the system's design which
form part of its non-functional requirements, in particular the requirements
of the software architecture on which the application's functional aspects are
developed.This includes:
\begin{itemize}
\item The architectural scope.
\item Quality requirements.
\item The integration and access channel requirements.
\item The architectural constraints.
\item Architectural patterns and styles used.
\item The architectural tactics and strategies used.
\item The use of reference architectures and frameworks.
\item Access and integration channels.
\item Technologies used.
\end{itemize}
\section{Architecture requirements}
\subsection{Architectural scope}
We used Firebase as our persistent infrastructure to store the uploaded media, its meta data and the hash values that will be used for cryptography purposes.  The sending and receiving of files will be handled using REST API which is also a component of Firebase.  Firebase uses object relational mapping which allows us to store the potential digital evidence as objects and not in relations.

\subsection{Quality requirements}
	\begin{itemize}
	\item Security(core)
	\begin{itemize}
		\item The system must detect attacks from unwanted and unauthorized users.
		\item The system must resist attacks from unwanted and unauthorized users.
		\item The system must recover from attack from unwanted and unauthorized users.
		\item The system must minimize access and permissions given to users who do not have the
			required privileges.
		\item All communication of sensitive data must be done securely through
			encryption and secure channels.
	\end{itemize}
	\item Scalability(core)
	\begin{itemize}
		\item The system operate effectively under a load of registered users.
		\item The system scales out resources.
		\item The system manages resource demand.
	\end{itemize}
	\item Reliability(core)
		\begin{itemize}
			\item The system must prevent faults.
			\item The system must detect faults.
			\item The system must recover from faults.
		\end{itemize}
	\item Integrability(core)
		\begin{itemize}
			\item The system must integrate with law enforcement server.
			\item The system integrate with Google maps for Geo-location.
		\end{itemize}
	\item Performance(core)
		\begin{itemize}
			\item Throughput:The rate at which incoming requests are completed.
			\item The system minimize the responsive time.
		\end{itemize}
	\item Usability(core)
		\begin{itemize}
			\item The system must be efficient.
			\item The system must be easy to use.
			\item The user must be satisfied by the system.
		\end{itemize}
	\item Deployability(important)
		\begin{itemize}
			\item The system is deployed into cross platforms.
		\end{itemize}
	\end{itemize}
\subsection{Integration and access channel requirements}
The system will be accessed using Mobile/Android Applications clients through the use of ngCordova,  browser clients which is where ionic's web based framework comes in and web services clients.
\subsection{Architectural constraints}
\begin{itemize}
\item There are multiple factors that may place constraints on the architecture that
is being developed for the uWatch System

\item This is developed under ionic framework which is simply web based, thus it put constraints on certain programming languages Compliance with existing standards
\end{itemize}
\section{Architectural pattern or styles}
	\begin{itemize}
		\item MVC(Model-View-Controller)
			\begin{itemize}
				\item Ionic framework is build from this pattern, where the view is the template HTML pages, the model is the JSON object to and from the Firebase database and the controller is the JavaScript functions to implement the system's functionality.
			\end{itemize}
		\item Pipes and Filters
		\begin{itemize}
		\item We will be tagging images, videos and audio files being uploaded to the Firebase and using this architectural pattern to allow us to easily search and filter by tags and categorize the PDE being uploaded.
		\end{itemize}
		\item Client/Server 
			\begin{itemize}
				\item uWatch allows multiple clients to access the system using N-tier
				architectural style.The benefit of using N-tier architectural style is that
				it improves the scalability of the system.
				\item The server side is the back-end of the system which manages the centralized data and access to the database from Firebase. The aim of
having the back-end manages the data is to achieve higher security. 
			\end{itemize}
	\end{itemize}
\section{Architectural tactics or strategies}
\begin{itemize}
\item Security: We do not cache user credentials nor keep them in history. This forces users to login everytime they need to login to the our system.  Users have a limited amount of tries to login. 
\item Maintainability and Flexibility: Our views are templated, separating the view from the templates. Since our system relies on AngulaJS, we inject dependencies and as such the is Inversion of control (IoC).
\item Perfomance: We catch users' retried PDE so that the system can respond faster especially on slow internet connection. Firebase DB also caches data in case the DB goes down and once it goes up again, it synchronizes the cached data with the data in the DB.
\item Scalability: We decouple the client from the serve and only have a handle to the Firebase serve in a form of a link and connecting to its API, this is possible through a RESTful API.
We use local storage to store any PDE that the user chooses to upload later in case the user did not have internet connection at the time the PDE was captured.
\item Reliability and Availability: caching and local storage as explained above.
\item Auditability - Users login before they can upload a PDE.
	- We use date, time-stamps and Geo-location.
	- Data that is send to and from the serve is encrypted.
\item Testing: we will implement unit testing and integration tests to see if the pre and post condition are/were met.
\item Usability: We use nice icon and buttons and less text and encourage users who can not read to be able to use our system. Since it is a web base, we develop our system in away that blind and visual impaired users will be able to use screen readers to access our services.
\item Integrability: Our system is integratable since it is decoupled form the cloud/database, it can work with any database or a cloud for that matters or simply a local storage. 
\item Deployability: The system is cross platform, it will be deployable on IOS, Android, Windows, Amazon and blackberry platforms.	
\end{itemize}
\section{Use of reference architecture and frameworks}
\begin{itemize}
	\item We are using Ionic framework to implement our mobile application. Ionic framework is a mobile base 			framework build on the following frameworks:
	\begin{itemize}
		\item ngCordova – An Apache cordova framework which packages HTML, css and js files in to respective platform such as Android, IOS, Blackberry, Windows mobile etc.
		\item AngulaJS – Is a Google JavaScript framework that helps implement controllers and services 				for our project.
		\item Bower and Node package manager (NPM) to help install dependencies and plugins into our 		project in order for us to extend functionality beyond your normal JS capabilities.
		\item UI-Routes – we use routes to changes from one view to the other and from one state to another, this framework help us achieve exactly that.
		\item Gulp js – for simplifying our project and stripping out all unnecessary comments on our codes so that it can compile faster and respond quicker. Furthermore, the packaged final project file is relatively smaller and making it easy to deploy, share or to store.
		\item  SASS – this is the framework that deals with the look and feel of our application. SASS is described as css on steroid on the sass website.
	\end{itemize}
\item Reasons for choosing ionic framework:
	\begin{itemize}
		\item Since it is a web based, it leverages our web development skills and the knowledge we have of the web based languages such as js, HTML and css.
		\item It is cross platform, one project will be deployed on every platform. Meaning we do not need to write special code to accommodate a specific platform or learn the native language that each platform uses. This speeds up production and is convenient since we can not assume that users will be using one specific platform.
\item It is pretty powerful. We use ngCordova’s plugins to access native device camera, video camera, audio recording, media file, local storage etc… All this is done effortless.
\item Our app has to access one’s Geo-location and it is relatively easy to that with ionic than will a native language.
\item Is easy to implement security features in javascript
\item Is easier to debug, you can also debug in a browser’s console.
\item It uses less resources, you do not need an emulator to test features such as Geo-location or encryption
\item There are many encryption js libraries. For example forgejs, cryptojs and they both implement AES cryptography.
\item It's easy to work with JSON object.
\item It has a great community and up to date plugins.
	\end{itemize}
\end{itemize}
\section{Access and integration channels}
\begin{itemize}
\item Integration channels that we using:
\begin{itemize}
\item We using a REST API which uses the Library cURL(which is a computer software project providing a library and command-line tool for transferring data using various protocols)

\item This REST API uses : 
\begin{itemize}
\item HTTPS (Hypertext Transfer Protocol Secure) is a protocol that allows safe and secure transfer of data over a network.
\item HTTP PUT for retrieving, writing and updating of data.
\end{itemize}

\item API specifications used in the integration of the systems
\begin{itemize}
\item Authentication API - which is an API that simply tells the Firebase cloud storage who the user is
\item Security API - which an API that specifies rules(read,write and validate),what the client can be able to access and what restricted to them and it provides encryption strategies,for example Forge(which is an encryption method).
\end{itemize}

\item Quality requirements that can be achieved through the implementation of the access channels mentioned.
\begin{itemize}
\item Reliability so that the system is online as much as possible and that data transfer isn’t easily corruptible and that the transfer is fast.(because Firebase the client make updates local even if there's internet latency or completely offline, firebase will synchronize everything once the network is back)

\item Security of user data being transferred.

\item Scalable so that a large amount of users data can be transferred to and from the systems.

\item Maintainable, the integration with the system must be easily maintained. (because for example security rules are all defined in one place.
\end{itemize}
\end{itemize}

\section{Technologies}
Technologies Used in the development of our Application are stated below along with why the choice was made and it's benefits.
\begin{itemize}
\item Ionic framework
 \begin{itemize}
\item Ionic framework is a great framework that supports hybrid application which has many benefits, specifically in terms of access to third party code , speed development and  platform support.
\item It is an Model-View-Controller framework which separates concerns, allowing re-use of the business logic; it enables parallel development by separate teams which is ideal when working in a project such as this one.  The MVC framework uses AngularJS for controllers and services, HTML for view and JSON for the model which are all popular and easy to learn languages.
 \end{itemize}
\item ngCordova
\begin{itemize}
\item ngCordova is a collection of 63+ AngularJS extension on top of the cordova API that make it easy to build, test and deploy Cordova applications with AngularJS,  ngCordova uses packages and libraries, compiles source code and deploys the source code to Apk (android), Xap (windows mobile) and ipa (ios).
\end{itemize}
\item Firebase
\begin{itemize}
 \item Firebase is a Google cloud server which powers the applications back-end, this covers data storage, user authentication, static hosting and more.  The data will be stored as a JSON string and synchronized to every connected client.  Firebase easily authenticates users across all platform SDK in just a few lines of code.  It uses Object relational mapping and and REST API.
\item Security provided by Firebase includes:
\end{itemize}
\item Authentication API-who the user is.
\item Support SSL on all clients
\item Uses Bcrypt for password storage
\item Uses the JSON Web Token for standard credentials
\item Generate Server-Signed Tokens
\item All security Logic is put in one pace
\end{itemize}
\item The benefits of Firebase:
\begin{itemize}
\item Scalability-operations occur local/client side
\item flexibility- No meta-data to manage
\end{itemize}
\end{itemize}

\section{References}
\begin{itemize}
\item All About Ionic (author and date unknown) 
Available from:http://ionicframework.com/docs/guide
[Accessed: 31 july 2015]
\item Why use ngcordova (author and date unknown) 
Available from:http://stackoverflow.com/questions/31011051
[Accessed: 31 july 2015]
\item Features  (author and date unknown) 
Available from:https://www.firebase.com/features.html 
[Accessed: 31 july 2015]
\end{itemize}
\end{document}

